\documentclass{beamer}

\usepackage[utf8]{inputenc}
\usepackage[turkish]{babel}

\title{Pardus}
\author{S.Çağlar Onur, caglar@pardus.org.tr}
\institute{
    Pardus Projesi [TÜBİTAK / UEKAE]\\
    Linux Kullanıcıları Derneği}
\usetheme{Warsaw}

\pgfdeclareimage[width=5cm]{pardus-logo}{pardus-logo}
\pgfdeclareimage[height=3cm]{evrim}{evrim}
\pgfdeclareimage[width=4cm]{pisi}{pisi}
\pgfdeclareimage[width=4cm]{farkli}{farkli}
\pgfdeclareimage[width=4cm]{yali}{yali}
\pgfdeclareimage[width=4cm]{tasma}{tasma}
\pgfdeclareimage[width=4cm]{kaptan}{kaptan}

\begin{document}

\frame{\titlepage}

\frame{
    \frametitle{İçerik}
        \begin{columns}
            \begin{column}{5cm}
                \tableofcontents
            \end{column}

            \begin{column}{5cm}
                \pgfuseimage{pardus-logo}
            \end{column}
        \end{columns}
}


\section{Tarihçe}

\frame<beamer>{\tableofcontents[current]}

\subsection{Neden?}
\frame{
    \frametitle{Neden?}
        \only<1>{
            \begin{block}{Ana sözleşme}
            Pardus, UEKAE tarafından, bilişim okur-yazarlığına sahip bilgisayar kullanıcılarının temel masaüstü ihtiyaçlarını hedefleyerek; mevcut Linux dağıtımlarının üstün taraflarını kavram, mimari ya da kod olarak kullanan; otonom sisteme evrilebilecek bir yapılandırma çerçevesi ve araçları ile kurulum, yapılandırma ve kullanım kolaylığı sağlamak üzere geliştirilen bir GNU/Linux dağıtımıdır
            \end{block}
        }
        \only<2>{
            \begin{itemize}
                \item Bağımsızlık, güvenlik ve tasarruf amacıyla, kritik uygulamaların üzerinde çalışabileceği, standartlara uygun şekilde kaynak kodu açık olan ve finansal yük oluşturmadan yaygınlaştırılabilecek bir işletim sistemi
                \item Türkiye’nin bilgi teknolojileri konusundaki etkinliğinin katma değerli projelere yöneltilmesi, araştırma ve geliştirme ağırlıklı yüksek teknoloji üretimi
                \item Ülke gereklerine bağlı olarak teknolojik gelişmenin yönünü belirlemek, farklı alanların ağırlığını değiştirmek ve dolayısıyla söz konusu işletim sisteminin yol haritasına hakim olmak
            \end{itemize}
        }
        \only<3>{
            \begin{itemize}
                \item Tam Türkçe desteğinin, kullanıcıya görünen tüm mesaj ve belgelerin Türkçe olması yoluyla sağlaması
                \item Mevcut Linux dağıtımlarından ve rakip diğer işletim sistemlerinden daha kolay kurulabilen ve kullanılabilen bir işletim sistemi olması
                \item Diğer Linux dağıtımlarının üstün yanlarını alarak
                \item Araç temelli ve teknoloji merkezli bir tasarım yerine görev temelli ve insan merkezli bir yaklaşımla ve esnekliği ve yüksek performansı sağlayabilecek modüler bir yapıda tasarlanması
            \end{itemize}
        }
}

\subsection{Nasıl?}
\frame{
    \frametitle{2007'ye doğru}
        \begin{itemize}
            \item \textbf{2003}: Kavramsal hazırlık. "Neden?", "Nasıl?", "Kim ile?" sorularına yanıtlar
            \item \textbf{2003 sonu}: \emph{"Dağıtım yapacağız"}
            \item \textbf{2004 ilk çeyrek}: Geliştirici ekibin toplanması
            \item \textbf{2004}: Teknik analiz ve alt yapının hazırlanması
            \item \textbf{2005 ilk çeyrek}: Pardus Çalışan CD (1 Şubat)
            \item \textbf{2005 sonu}: Pardus 1.0 (26 Aralık)
            \item \textbf{2006 sonu}: Pardus 2007 (18 Aralık)
        \end{itemize}
        \center{\pgfuseimage{evrim}}
}

\section{Pardus Yenilikleri}

\frame<beamer>{\tableofcontents[current]}

\subsection{Türkçe çözümleri}
\frame{
    \frametitle{Türkçe kullanmak istiyoruz}
        \begin{itemize}
            \item Yerelleştirme çalışmaları
            \item Bireysel çabalar
            \item Sesimizi, istediğimiz kadar, duyurmak için yeterli olmamışlardı
            \item Sorunları buluyor ve çözüm üretiyoruz
            \item Çözümlerimizi herkes ile paylaşıyoruz
        \end{itemize}
}

\frame{
    \frametitle{İmla denetimi}
        \begin{itemize}
            \item Masaüstünde, her yerde Türkçe imla denetimi yapabilmek
            \item Türkçe yazabilmek, yazılanı okuyabilmek
            \item ı-I, i-İ harflerinin dönüşümleri yüzünden uygulama hatalarına takılmak
            \item Artık hayal değil ;)
        \end{itemize}
}

\subsection{PiSi}
\frame{
    \frametitle{PiSi}
        \begin{columns}
            \begin{column}{6cm}
            \only<1>{
                \begin{itemize}
                    \item \emph{Packages Installed Successfully as Intended}
                    \item Ne "Paket" ne de "Paket yönetimi" yeni değil kavramlar değil!
                    \item Doğru yapılması gereken bir işi doğru yap!
                    \item Paket yönetimi dağıtım için \textbf{çok önemli}
                    \item Parçaları bir araya getiren sistem
                \end{itemize}
            }
            \only<2>{
                \begin{itemize}
                    \item Daha kolay paket \emph{yönetimi}
                        \begin{itemize}
                            \item Geniş özellik kümesi, sorunsuz kurulum ve güncelleme
                            \item Grafiksel, tam bir yönetim arayüzü
                        \end{itemize}
                    \item Daha kolay paket \emph{üretimi}
                        \begin{itemize}
                            \item Genişletilebilir, üzerine kolayca yeni uygulamalar eklenebilir bir sistem
                            \item Geliştiriciler zamanlarını paket \emph{yapmaya} değil, paket sorunlarını \emph{sorunlarını} çözmeye harcamalı
                        \end{itemize}
                \end{itemize}
            }
            \only<3>{
                \begin{itemize}
                    \item Yüksek seviyeli ve düşük seviyeli paket yönetim özellikleri
                    \item Kaynak tabanlı ve ikili paket sistemlerinin iyi huyları
                    \item Paketler XML dosyaları ve basit python betikleri ile ifade ediliyor
                    \item Paketler bileşen ve kategoriler ile düzenleniyor
                    \item Python ile yazılmış
                    \item İkili paketler LZMA ile sıkıştırılmış PKZIP arşivleri
                \end{itemize}
            }
            \only<4>{
                \begin{itemize}
                    \item Şu an için resmi depoda 1221 paket
                    \item GNOME/XFCE/medya oynatıcıları v.b.'nin bulunduğu contrib deposu
                    \item Paket yapmak, paket bakımı yapmak çok kolay
                    \item Benzerlerinden kat ve kat daha ufak bir kod, fazlaca özellik
                \end{itemize}
            }
            \only<5>{
                \begin{itemize}
                    \item Derleme çiftliği!
                    \item SVN ile entegre çalışma
                    \item Paketler depoya girer girmez derlenip PiSi paketi haline geliyorlar
                    \item Benzerlerinden kat ve kat daha ufak bir kod
                \end{itemize}
            }
        \end{column}

        \begin{column}{4cm}
            \pgfuseimage{pisi}
        \end{column}
    \end{columns}
}


\subsection{ÇOMAR}
\frame{
    \frametitle{ÇOMAR}
        \center{\pgfuseimage{farkli}}
        \only<1>{
            \begin{itemize}
                \item Yeni bir yaklaşım
                \item \emph{Herkes onu arıyor!}
                \item Sorunun doğru tarifi $\to$ Doğru çözüm
            \end{itemize}
        }
        \only<2>{
            \begin{itemize}
                \item Yapılandırma arayüzleri görev tabanlı olmalı
                \item Gündelik işler için komut satırı gerekmemeli
                \item \emph{Herkes belge okumuyor}
                \item Bilgisayar kendi işini kendi yapmalı
                \item Uygulamalar bir arada çalışabilmeli
            \end{itemize}
        }
        \only<3>{
            \begin{itemize}
                \item Şu anda neler yapabiliyor
                \begin{itemize}
                    \item Açılış sistemi; \textbf{hızlı}, kolay
                    \item Yazılım kurulum ve güncelleme (PiSi ile birlikte)
                    \item Basit ve hızlı, profil tabanlı ağ yapılandırması
                    \item Otomatik grafik arayüz yapılandırması
                    \item Kullanıcı yönetimi
                \end{itemize}
            \end{itemize}
        }
        \only<4>{
            \begin{itemize}
                \item Önümüzdekiler
                \begin{itemize}
                    \item Sunucu yönetimi
                    \item Uzaktan yönetim
                    \item Güç yönetimi
                    \item Hayal gücümüz!
                \end{itemize}
            \end{itemize}
        }
}

\subsection{YALI}
\frame{
    \frametitle{YALI}
        \center{\pgfuseimage{yali}}
        \only<1>{
            \begin{itemize}
                \item "Kurulum Yazılımı" yeni değil
                \item \emph{Herkes için kolay ve hızlı bir kurulum deneyimi}
                \item Sade, basit ve görev temelli ve insan odaklı
            \end{itemize}
        }
        \only<2>{
            \begin{itemize}
                \item Bir kurulum yazılımı sadece kendi işlevlerini yapmalı
                \item Pisi ve ÇOMAR ile entegre
                \item Ufak, hızlı
            \end{itemize}
        }
}

\subsection{Kaptan}
\frame{
    \frametitle{Kaptan}
        \center{\pgfuseimage{kaptan}}
        \begin{itemize}
            \item Kullanıcının ilk gördüğü uygulama
            \item Kullanıcının masaüstü görüntüsünü, ağ ayarlarını, paket yöneticisi ile ilgili ayarları yaptığı uygulama
            \item Basit, sade ve amaca yönelik
        \end{itemize}
}

\subsection{Tasma}
\frame{
    \frametitle{TASMA}
        \center{\pgfuseimage{tasma}}
        \begin{itemize}
            \item Kullanıcının sisteme dair ayarları değiştirebildiği uygulama
            \item Bir kısmını KDE araçları diğer kısmını Pardus yapılandırma araçları oluşturuyor
            \item Basit, sade ve amaca yönelik
        \end{itemize}
}

\subsection{Pardusman}
\frame{
    \frametitle{Pardusman}
        \begin{itemize}
            \item Sürüm çıkartmak hiç bu kadar kolay olmamıştı :)
            \item Benzer araçlar yanında ufak kalıyor :P
            \item Çalışan/Kurulan/Yazan/Uçan CD'ler bir tık ile önümüzde
            \item Herkes kendine özel Pardus yapabilir!
        \end{itemize}
}

\subsection{Diğer Yenilikler}
\frame{
    \frametitle{Diğer Yenilikler}
        \begin{itemize}
            \item 70 dil desteği (Türkçe, İngilizce, Almanca, Hollandaca ve İspanyolca kurulum)
            \item AppArmor güvenlik yazılımı
            \item Gelişmiş donanım tanıma sistemi (Müdür)
            \item Gelişmiş güç yönetimi (Suspend/Hibernate v.b.)
            \item Gelişmiş X yapılandırma aracı (Zorg)
            \item Güncel paketler/teknolojiler [bkz: AIGLX :)]
            \item Masaüstü arama yazılımları (Beagle/Kerry)
            \item Ntfs-3g ile NTFS disklere yazma desteği
            \item Onlarca yeni oyun (Çok güzeller :P)
            \item Suse'nin gelişmiş Kickoff menüsü
            \item Xen ve Qemu gibi sanal makina yazılımları
        \end{itemize}
}

\section{Sıklıkla Sorulan Sorular}

\frame<beamer>{\tableofcontents[current]}

\frame{
    \frametitle{Sıklıkla Sorulan Sorular}
        \begin{itemize}
            \item Neden Debian, Fedora, (favori dağıtımım) değil?
            \item Neden RPM, DPKG, (favori paket yöneticim) değil?
            \item Neden Qt/Neden KDE/Neden Python?
            \item Neden TÜBİTAK bu işi yapıyor?
            \item TÜBİTAK projeye desteğini çekerse ne olur?
            \item Kimler kullanıyor/kullanacak?
            \item Pardus Sunucu? Pardus Gömülü? Pardus Medya Merkezi?
            \item Pardus'da çalışmak/staj yapmak istiyorum!
        \end{itemize}
}

\frame{
    \frametitle{Mutlu son...}
        \begin{centering}
            \Huge{\alert{Teşekkürler}} \newline
        \end{centering}

        \begin{block}{İLETİŞİM}
            \begin{itemize}
                \item E-Posta: bilgi@pardus.org.tr
                \item Web: http://www.pardus.org.tr
                \item E-Posta Listeleri: http://liste.uludag.org.tr/mailman/listinfo
                \item SVN Depoları: http://svn.pardus.org.tr/
            \end{itemize}
        \end{block}
}

\end{document}
