\documentclass{beamer}

\usepackage[utf8]{inputenc}
\usepackage[turkish]{babel}

\usepackage{beamerthemesplit}
\title{COMAR: A new approach to System Management}
\author[Gökmen Göksel]{gokmen@pardus.org.tr}
\date{Brussels, Belgium\\ February 05, 2011}
\institute{
  Center of Research For Advanced Technologies Of Informatics And Information Security\\
  TÜBİTAK/BİLGEM}

\setbeamertemplate{navigation symbols}{}
\usetheme{Warsaw}
\usecolortheme{seagull}

%images
\pgfdeclareimage[interpolate=true,width=8cm]{kde_integration}{kde_integration}
\pgfdeclareimage[interpolate=true,width=7cm]{usermanager}{usermanager}
\pgfdeclareimage[interpolate=true,width=7cm]{polkit}{polkit}
\pgfdeclareimage[interpolate=true,width=10cm]{sync}{sync}
\pgfdeclareimage[interpolate=true,width=10cm]{oldstate}{oldstate}
\pgfdeclareimage[interpolate=true,width=5cm]{newstate}{newstate}
\pgfdeclareimage[interpolate=true,width=5cm]{mudur}{mudur}
\pgfdeclareimage[interpolate=true,width=2cm]{pardus-logo}{logo}
\logo{\pgfuseimage{pardus-logo}}

\begin{document}

\frame{\titlepage}

\section{GNU/Linux System Management}

\subsection{Basics}
\frame
{
    \frametitle{Basics}
    \begin{itemize}
       \item Package Management
       \item Service Management
       \item Network Management
       \item Boot Management
       \item Disk Management
       \item User Management
       \item Firewall Management
       \item ...
    \end{itemize}
}

\subsection{Current Problems}
\frame
{
    \frametitle{Current Problems}
    \only<1>{
    \begin{enumerate}
     \item Each library has different API and it is not easy to make universal interfaces for the same purposes
     \item GUI or CLI code includes operational code so its hard to use operational code from different apps
     \item Application base authentication (GUI or Cli) and it is not possbile to set privileges atomically
     \item Different apps can modify same data and syncronization between these apps is not an easy issue to solve
    \end{enumerate}
    }
    \only<2>{
    \begin{figure}
     \begin{center}{\pgfuseimage{oldstate}}\end{center}
    \end{figure}
    }
}

\section{Pardus Solution: COMAR}

\subsection{What is COMAR ?}
\frame
{
    \frametitle{What is COMAR ?}
    \only<1>{
    \begin{block}{What is COMAR ?}
     Çomar (pronounced chow-mar), is the COnfiguration MAnageR that helps
the installed software operate flawlessly. Çomar knows the tasks 
that can be provided by each application, together with the 
functionality they depend on and other information. Different
applications may adapt themselves according to the presence and 
capability of their peers.
    \end{block}
    }
    \only<2>{
    \begin{figure}
     \begin{center}{\pgfuseimage{newstate}}\end{center}
    \end{figure}
    }
}

\section{COMAR Features}

\subsection{Shared Models}
\begin{frame}[fragile]
    \frametitle{Shared Models}
    \scriptsize
    \begin{verbatim}
<comarModel>
  <interface name="Disk.Manager">
    <method name="getDevices" access_label="get">
      <arg name="devices" type="as" direction="out"/>
    </method>
    <method name="mount" access_label="mount">
      <arg name="device" type="s" direction="in"/>
      <arg name="path" type="s" direction="in"/>
    </method>
    <method name="umount" access_label="mount">
      <arg name="device" type="s" direction="in"/>
    </method>
    ...
  </interface>
</comarModel>
    \end{verbatim}
\end{frame}

\subsection{Different scripts for same model}
\begin{frame}[fragile]
    \frametitle{Different scripts for same model}
    \begin{figure}
     Each package can provide different script for the same model
     \begin{center}{\pgfuseimage{mudur}}\end{center}
    \end{figure}
\end{frame}

\subsection{Language independent usage}

\begin{frame}[fragile]
    \frametitle{Using D-Bus}
      \begin{verbatim}
~ > qdbus --system tr.org.pardus.comar 
          /package/mudur 
          tr.org.pardus.comar.Disk.Manager.getDevices

/dev/sda
/dev/sdb
/dev/sdc
      \end{verbatim}
\end{frame}

\begin{frame}[fragile]
    \frametitle{Using Python}
      \begin{verbatim}
import comar 
link = comar.Link()
link.Disk.Manager["mudur"].getDevices()

dbus.Array([dbus.String(u'/dev/sda'), ..))
      \end{verbatim}
\end{frame}
\begin{frame}[fragile]
    \frametitle{Using Shell}
      \begin{verbatim}
~ > hav call mudur Disk.Manager getDevices
 
dbus.Array([dbus.String(u'/dev/sda'), ..))
      \end{verbatim}
\end{frame}

\subsection{Authentication using Polkit}

\begin{frame}[fragile]
    \frametitle{Provides atomic privileges}
      \scriptsize
      \begin{verbatim}
~ > cat /usr/share/polkit-1/actions/tr.org.pardus.comar.user.manager.policy 
...
<policyconfig>
    <vendor>Pardus</vendor>
    <vendor_url>http://www.pardus.org.tr</vendor_url>
    <icon_name>system-users</icon_name>

    <action id="tr.org.pardus.comar.user.manager.adduser">
        <description>Add user</description>
        <message>System policy prevents adding new user</message>
        <defaults>
            <allow_any>auth_admin_keep</allow_any>
            <allow_active>auth_admin_keep</allow_active>
            <allow_inactive>auth_admin_keep</allow_inactive>
        </defaults>
    </action>
    ...
</policyconfig>
      \end{verbatim}
\end{frame}

\begin{frame}[fragile]
    \frametitle{COMAR call Polkit agent when necessary}
     \begin{figure}
      \begin{verbatim}~ > hav call openssh System.Service stop\end{verbatim}
     \begin{center}{\pgfuseimage{polkit}}\end{center}
    \end{figure}
\end{frame}

\begin{frame}[fragile]
    \frametitle{Managing privileges over User Manager}
     \begin{figure}
     \begin{center}{\pgfuseimage{usermanager}}\end{center}
    \end{figure}
\end{frame}

\subsection{Applications are synced}

\begin{frame}[fragile]
    \frametitle{When you change a service state everybody knows it}
    \begin{figure}
     \begin{center}{\pgfuseimage{sync}}\end{center}
    \end{figure}
\end{frame}

\section{Current State}
\subsection{System Management in Pardus 2011}
\begin{frame}[fragile]
    \frametitle{System Management in Pardus 2011}
    \begin{figure}
     \begin{center}{\pgfuseimage{kde_integration}}\end{center}
    \end{figure}
\end{frame}

\frame
{
    \frametitle{Thanks}
    \begin{itemize}
        \item Questions ?
	\item Demo ?
    \end{itemize}

    \begin{itemize}
        \item Home Page : http://www.pardus.org.tr
        \item Developer Page : http://developer.pardus.org.tr
        \item E-Mail Lists : http://lists.pardus.org.tr
        \item Bugzilla : http://bugs.pardus.org.tr
        \item Wiki : http://en.pardus-wiki.org
    \end{itemize}

}

\end{document}
