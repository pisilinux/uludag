\documentclass[a4paper,10pt]{article}
\usepackage[left=1cm,top=2cm,right=2cm,bottom=1cm]{geometry}
\usepackage[utf8x]{inputenc}

\title{Paket Yöneticisi Test Aşamaları}
\author{Semen Cirit}

\renewcommand{\labelenumi}{\arabic{enumi}.}
\renewcommand{\labelenumii}{\arabic{enumi}.\arabic{enumii}.}
\renewcommand{\labelenumiii}{\arabic{enumi}.\arabic{enumii}.\arabic{enumiii}.}
\renewcommand{\labelenumiv}{\arabic{enumi}.\arabic{enumii}.\arabic{enumiii}.\arabic{enumiv}.}

\begin{document}

\maketitle

\begin{enumerate}
    \item Menüden paket yöneticisini seçiniz.
	
	  Sorunsuz bir şekilde açılabildiğini gözlemleyin.
 

    \item Paket yöneticisi menü çubuğu:

    Sol tarafta bulunan kutu ilgili paket gruplarını göstermektedir. Bu paket grupları depolarımızda bulunan ilgili bileşenlerin paketlerini listelemektedir.

    Örnek olarak aşağıda verilecek olan grup ve bileşen eşleşmeleri aşağıda gerçekleştirilecek testcaseler için gereklidir.
     \begin{table}[h]
	  \centering
	  \begin{tabular}{|c|c|}
		  \hline
		  Grup & Bileşen \\
		  Virtualization & hardware.virtualization \\
		  Servers & server.database \\
		  \hline
	  \end{tabular} 
	  \label{tab:tbl}
    \end{table}
    
    \begin{enumerate}
        \item Dosya menü çubuğu
    
        Düşen kutu seçenekleri
        \begin{enumerate}
            \item Kurulabilir Paketleri gösteri seçiniz.
            
	     Daha sonra yukarıda verilmiş olan bileşenlerden birini kullanarak aşağıdaki komutu çalıştırınız.
            \begin{verbatim}
            $ pisi la -Uc <seçtiğiniz bileşen>
            \end{verbatim}
            
	    Paket yöneticisindeki ilgili grubun konsol çıktısından alınan paket adlarını içerdiğini gözlemleyin.

            \item Kurulu Paketleri Gösteri seçiniz.

             Daha sonra yukarıda verilmiş olan bileşenlerden birini kullanarak aşağıdaki komutu çalıştırınız.
            \begin{verbatim}
            $ pisi li -c <seçtiğiniz bileşen>
            \end{verbatim}
            Paket yöneticisindeki ilgili grubun konsol çıktısından alınan paket adlarını içerdiğini gözlemleyin.
    
            \item Güncellemeleri gösteri seçiniz.

             Daha sonra yukarıda verilmiş olan bileşenlerden birini kullanarak aşağıdaki komutu çalıştırınız.
            \begin{verbatim}
            $ pisi lu -c <seçtiğiniz bileşen>
            \end{verbatim}
            Paket yöneticisindeki ilgili grubun konsol çıktısından alınan paket adlarını içerdiğini gözlemleyin.
    
            \item Çık seçeneğini seçiniz. 
            Observe the situation.
        \end{enumerate}
    
        \item Ayarlar menü çubuğu
        
	 Düşen kutu seçenekleri
	 \begin{enumerate}
            \item Araç çubuğunu gösteri disable ediniz.
	
	     Araç çubuğunun kaybolduğunu gözlemleyin.
            \item Durum çubuğunu gösteri disable ediniz.
  
	    Durum çubuğunun kaybolduğunu gözlemleyiniz.

            \item Kısa yolları yapılandır'ı seçiniz.

            \begin{enumerate}
                \item Çıkan pencereden bir eylem seçiniz.

                \item İlgili eylem üzerinde iken kısayol seçiniz.
                \begin{itemize}
                    \item Kısayolu değiştiriniz.

		    Bu kısa yolu kullanarak ilgili eylemin gerçekleştiğini kontrol ediniz.
                    \item Özel bir kısa yol veriniz

		    Bu kısa yolu kullanarak ilgili eylemin gerçekleştiğini kontrol ediniz.
                    \item Kısa yolu silmeyi deneyiniz.
		    
		    Sildiğiniz kısa yolu kullanarak ilgili eylemin gerçekleşmediğini kontrol ediniz.
		    
                \end{itemize}
		\item Öntanımlılara geri dön butonuna basınız.
      
		 Değiştirmiş olduğunuz bir kısayolun öntanımlı haline geri döndüğünü gözlemleyiniz.

		\item Eğer yazıcınız var ise print tuşuna basınız.
	  
		Çıktısının alınabildiğini gözlemleyiniz.
            \end{enumerate}
            \item Araç çubuğunu yapılandır'ı seçiniz.
            \begin{enumerate}
                \item Araç çubuğuna bir eylem ekleyiniz
		 
		Bu eylemin araç çubuğuna eklendiğini gözlemleyin.

                \item Araç çubuğundan bir eylem siliniz.
      
		Bu eylemin araç çubuğundan silindiğini gözlemleyin.

                \item Bir eylemi yukarı taşıyın
      
		Eylemlerin araç çubuğunda uygun sırada olduğunu gözlemleyin.
                \item Bir eylemi aşağıya taşıyın.

                Bu eylemin araç çubuğunda uygun sırada olduğunu gözlemleyin.
            \end{enumerate}
            \item Paket Yöneticisi Ayarları'nı seçiniz.
            \begin{enumerate}
                \item Genel ayarları seçiniz.
                \begin{itemize}
                    \item Sadece masaüstü uygulamalarını gösteri seçiniz.

                    .kde4/share/config/package-managerrc dosyası içerisinde "ShowOnlyGuiApp" değişkeninin "True" olduğunu gözlemleyiniz.
		     Paket yöneticisindeki paketlere baktığınızda sadece masaüstü uygulamalarının paketlerinin kaldığını gözlemleyiniz.

                    \item Sistem çekmecesi ikonunu etkinleştiri seçiniz.

                    .kde/share/config/package-managerrc dosyasında "SystemTray" değişkeninin "True" olduğunu gözlemleyin.
		     Paket yöneticisi ikonunun sistem çekmecesinde belirdiğini gözlemleyin.

                    \item Aralıklı güncelleme kontrolünü etkinleştiriniz. 
                 
                        .kde4/share/config/package-managerrc dosyasında "UpdateCheck" değişkeninin "True" olduğunu gözlemleyin.
                        
		    \item Gözlemleyebileceğiniz bir zaman aralığı veriniz. 
	
			 Bu verdiğiniz zaman aralığının .kde4/share/config/package-managerrc dosyasında "UpdateCheckInterval" değişkenine atanmış olduğunu gözlemleyin.    

                    \item band genişliği limitini kullanınız ve uygun bir aralık veriniz.

		     /etc/pisi/pisi.conf dosyasında "bandwith\_limit" değişkeninin verdiğiniz değişken ile aynı olduğunu gözlemleyiniz.

                     Bir paket indirerek paket indirme hızının verdiğiniz hız olduğunu, paket kuruluyor popup penceresinden gözlemleyiniz.
    
                \end{itemize}
                \item Önbelleği seçiniz.
                \begin{itemize}
                    \item Güncelleme kontrolü 
                    \begin{itemize}
                        \item indirilen yazılımlar için sabit disk önbelleği kullanı seçiniz.

                        /etc/pisi/pisi.conf dosyasında \texttt{package\_cache = True} olduğunu gözlemleyin.

                        \item Azami önbellek boyutuna uygun bir değer verin.

                        /etc/pisi/pisi.conf dosyasından bu verdiğiniz değerin "package\_cache\_limit" değişkenine atanmış olduğunu gözlemleyin.
                    \end{itemize}
                  
		      \item Önbelleği şimdi temizleyi çalıştırınız.

		      Aşağıdaki komutaları çalıştırınız: 
		      \begin{verbatim}
		      # cd /var/pisi
		      \end{verbatim}

		      "bash: cd: /var/pisi: Böyle bir dosya ya da dizin yok" gibi bir çıktı verdiğini gözlemleyiniz.
	    
		      \begin{verbatim}
		      # cd /var/cache/pisi/
		      \end{verbatim}
		
		      Eğer debug paketi kurmamış iseniz bu dizinin boş olduğunu, eğer kurmuş iseniz packages-debug dizininin bulunduğunu gözlemleyiniz.
		   
                \end{itemize} 
                \item Depolar

		 Komutunu çalıştırarak depolar ile ilgili yaptığınız işlemleri gözlemleyebilirsiniz.
		    \begin{verbatim}
		      # pisi lr
		    \end{verbatim}
                \begin{itemize}
                    \item Bir depo ekleyiniz.
		    
		    Deponun eklenmiş olduğunu yukarıdaki komutu kullanarak gözlemleyiniz.

		    Güncellemeleri göster deyip, eklediğiniz deponun güncellenip güncellenmediğini gözlemleyiniz.
                    \item Bir depo siliniz
		
		    Deponun silinmiş olduğunu yukarıdaki komutu kullanarak gözlemleyiniz.
                    \item Bir depoyu yukarı çekiniz
 
		     Deponun uygun yere gelmiş olduğunuyukarıdaki komutu kullanarak gözlemleyiniz.
                    \item Try to take down a repository 

                    Deponun uygun yere gelmiş olduğunuyukarıdaki komutu kullanarak gözlemleyiniz.
                \end{itemize}
                \item Vekil sunucu
                \begin{itemize}
                    \item  Vekil sunucu yoku seçiniz.

                    /etc/pisi/pisi.conf dosyasına bakın ve aşağıdaki değişkenlerin bu şekilde olduğunu gözlemleyin.
                    \begin{verbatim}
                    http_proxy = None
                    https_proxy = None
                    ftp_proxy = None
                    \end{verbatim}
    
                    \item Vekil sunucu kullanınız.
    
                    \begin{itemize}
                        \item Birer http, https ftp sunucusu ekleyiniz.
        
                        İlgili değişiklikleri /etc/pisi/pisi.conf dosyasından gözlemleyin.
        
                        \item Tüm protokoller için bu vekil sunucuyu kullan deyiniz. 
        
                        Tüm sunucular için verdiğiniz http sunucusu kullanıldığını gözlemleyiniz.
                    \end{itemize}
                \end{itemize}
                        \end{enumerate}
    \end{enumerate}
\item Yardım menü çubuğu
		
                \begin{itemize}
                    \item Uygulamanın dilini seç'i seçiniz 
                    \begin{itemize}
                        \item Birincil dili seçiniz
			  
			 Durumu gözlemleyiniz.
                        \item İkinci dili seçiniz.
		      
			 Durumu gözlemleyiniz.
                        \item İkincil dili siliniz
		  
			Durumu gözlemleyiniz.
                    \end{itemize}
                \end{itemize}

\end{enumerate}

    \item Araç çubuğu
    \begin{itemize}
        \item Araçlar çubuğuna sağ tıklayınız.

        \item Konumu değiştiriniz
      
	Araç çubuğunun seçtiğiniz konumda olduğunu gözlemleyiniz.
        \item Metin konumunu değiştiriniz.

	    Araç çubuğunda bulunan ikonların stillerinin seçmiş olduğunuz şekilde olduğunu gözlemleyiniz.
            
	    .kde/share/config/package-managerrc dosyasına seçmiş olduğunuz stilin "ToolButtonStyle" değişkenine atandiğını gözlemleyiniz.

        \item İkon boutunu değiştiriniz.
      
	Seçilmiş boyutun seçmiş olduğunuz şekilde olduğunu gözlemleyiniz.
        \item Araç çubuğunu kilitleyiniz.
    
            Ve tekrar sağ tıklayıp herhangi bir araç çubuğu özelliğini değiştirmeye çalışınız
	    
	    Hiçbir değişikliğin gerçekleşmediğini gözlemleyiniz.
    \end{itemize}


    \item Paket kurma
	
		
	İlgili paketin kurulu olup olmadığını aşağıdaki komut ile gözlemleyebilirsiniz:
	\begin{verbatim}
	 pisi info <paket-adı>
	\end{verbatim}

        \begin{itemize}
            \item Kurulabilir paketleri göster dedikten sonra. 
            \begin{itemize}
                \item Paketler listesinden bir paket seçip bu paketi kurunuz.
		
		Açılan pencerede pakete ait bağımlılıkların düzgün çıktığını gözlemlemleyiniz.	
	 	Note: Bu komut ile paket bağımlılıklarını Bağımlılıklar: başlığı altından gözlemleyebilirsiniz.
		\begin{verbatim}
		# pisi info <paketadı>
		\end{verbatim}
		Yukarıdaki komutu kullanarak paketin kurulmuş olduğunu gözlemleyiniz.
                \item Bu gruptaki tüm paketleri seç'e tıklayınız 

		Yukarıdaki komutu kullanarak listeden kurduğunuz birkaç paket için paketlerin kurulmuş olduğunu gözlemleyiniz.
  
            \end{itemize}
            \item Güncellemeleri göster dedikten sonra.
            \begin{itemize}
                \item Bir paket seçiniz ve güncellemeyi deneyiniz.

		Yukarıdaki komutu kullanarak paketin güncellenmiş olduğunu gözlemleyiniz.
                \item Bu gruptaki tüm paketleri seç'e tıklayınız

		Yukarıdaki komutu kullanarak listeden güncellediğiniz birkaç paket için paketlerin update olduğunu gözlemleyiniz.
            \end{itemize}
	\item Herhangi bir işlem yapılmakta iken çıkan kimlik doğrulama penceresine iptal değiniz.

	İptal işlemi sonucunda paket yöneticisinin sorunsuz bir şekilde yapılmakta olan işlemden önceki duruma geçtiğini gözlemleyiniz.

	\item Bir masaüstü uygulaması kurun ve parola sorma penceresinde parolamı anımsayı seçin.
		
		Bu paketi kurduktan sonra bu paketin ikonunu ve bilgisini içeren bir  pencere açıldığını gözlemleyin.

		Bu ikona tıklayın

		Uygulamanın sorunsuz bir şekilde açıldığını gözlemleyin.

	\item Parolamı hatırla aşaması bir önceki bölümde gerçekleştirilmişti.
  
	  Bu durumda paket kurmayı deneyin.

		Paket yöneticisinin parola sormadan paketi kurabildiğini gölemleyin.
        \end{itemize}
    \item Paket kaldırma
      \begin{itemize}
            \item Kurulu paketleri göster dedikten sonra.
            \begin{itemize}
                \item Paket listesinden bir paket seçip kaldırınız.
	  
		 Bu paketin kaldırıldığını yukarıdaki komutu kullanarak gözlemleyiniz.

                \item Bu gruptaki tüm paketleri seçiniz ve indiriniz.

		Yukarıdaki komutu kullanarak listeden sildiğiniz birkaç paket için paketlerin silinmiş olduğunu gözlemleyiniz.
            \end{itemize}
        \end{itemize}
    \end{enumerate}

\end{document}
