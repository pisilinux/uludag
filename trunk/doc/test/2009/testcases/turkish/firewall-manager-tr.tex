\documentclass[a4paper,10pt]{article}
\usepackage[turkish]{babel}
\usepackage[utf8]{inputenc}
\usepackage[left=1cm,top=2cm,right=2cm,bottom=1cm]{geometry}

\title{Güvenlik Duvarı Test Aşamaları}
\author{Semen Cirit}
\renewcommand{\labelenumi}{\arabic{enumi}.}
\renewcommand{\labelenumii}{\arabic{enumi}.\arabic{enumii}.}
\renewcommand{\labelenumiii}{\arabic{enumi}.\arabic{enumii}.\arabic{enumiii}.}
\renewcommand{\labelenumiv}{\arabic{enumi}.\arabic{enumii}.\arabic{enumiii}.\arabic{enumiv}.}

\begin{document}

\maketitle

\begin{enumerate}
    \item Menü $\rightarrow$ Sistem Ayarları yolunu izleyerek Güvenlik Duvarı Yöneticisi'ni açmayı deneyin.

    Sorunsuz bir şekilde açılabildiğini gözlemleyin.
    \item Menü $\rightarrow$ Uygulamalar $\rightarrow$ Sistem yolunu izleyerek Güvenlik Duvarı Yöneticisi'ni açmayı deneyin.

	Sorunsuz bir şekilde açılabildiğini gözlemleyin.
    \item Güvenlik Duvarı servisini başlat/durdur butonunu kullanarak durdurun.

    Şu komutu çalıştırın:

	\begin{verbatim}
	# service list
	\end{verbatim} 

    iptables servisinin kapalı olduğunu gözlemleyin.

    \item Güvenlik Duvarı servisini başlat/durdur butonunu kullanarak başlatın.

        Şu komutu çalıştırın:
	
	\begin{verbatim}
	# service list
	\end{verbatim}

       iptables servisinin başlatılmış olduğunu gözlemleyin.

\item Gelen Bağlantıları Engelle ve Giden bağlantıları engelle seçenekleri için, bu seçeneklerin sağ tarafında bulunan yapılandırma butonuna basın ve aşağıdaki işlemleri gerçekleştirin:

    \begin{enumerate}
        \item Port eklemeye çalışın.
        \item Port silmeye çalışın.
        \item Bir portu yukarı çekmeye çalışın.
        \item Bir portu aşağı çekmeye çalışın.
    \end{enumerate}

Bu işlemlerin sorunsuz bir şekilde gerçekleştiğini gözlemleyin.

\textbf{Not:} Bu komutun aşağıda gerçekleşen her durum için çıktısını gözlemleyin.

\begin{verbatim}
    # iptables --list-rules
\end{verbatim} 

\item Gelen Bağlantıları Engelle seçeneği:

 Güvenlik duvarı Yöneticisinden port ekledikten sonra.
\begin{enumerate}
  
        \item Gelen Bağlantıları Engelle seçeneğini aktifleştirin.
        
	İlgili komutun çıktısının şunları içerdiğini gözlemleyin:
	\begin{verbatim}
	-A PARDUS-IN-MOD-SERVING -p tcp -m multiport --dports <EklenenPort> \
	-j ACCEPT
	-A PARDUS-IN-MOD-SERVING -p tcp -m multiport --dports 0:1024 \
	-m tcp --tcp-flags FIN,SYN,RST,ACK SYN -j REJECT --reject-with \
	icmp-port-unreachable
	-A PARDUS-IN-MOD-SERVING -p udp -m multiport --dports 0:1024 \
	-j REJECT --reject-with icmp-port-unreachable
	\end{verbatim} 
        \item Gelen bağlantıları engelle şeçeneğini iptal edin:

              Yukarıdaki satırların ilgili komutun çıktısından silindiğini doğrulayın.
\end{enumerate}

\newpage

\item Internet paylaşımı seçeneği:

Internet paylaşımını aktifleştirin
    Internet paylaşımını aktifleştirin
    	(Dahili iki ethernet kartınız veya fazladan harici bir ethernet kartınız varsa bu bölümü test edebilirsiniz, yoksa bu adımı geçin.)
        \begin{enumerate}

        \item İnternet ve ev ağınız için farklı köprüler seçin.
  
         İlgili komutun çıktısının aşağıdakileri içerdiğini gözlemleyin:
	\begin{verbatim}
	-A PARDUS-FW-MOD-SHARING -i <input> -o <output> -m state \
	--state ESTABLISHED,RELATED -j ACCEPT
	-A PARDUS-FW-MOD-SHARING -i <output> -o <input> -j ACCEPT
	-t nat -A PARDUS-POST-MOD-SHARING -o <input> -j MASQUERADE
	\end{verbatim} 
        \item Aynı değerleri verin

    	Yukarıdaki satırların ilgili komutun çıktısından silindiğini gözlemleyin.
    \end{enumerate}

\item Giden bağlantılar engelle seçeneği: 

Yapılandırma kısmına bir port ekledikten sonra.
        \begin{enumerate}
        \item Giden bağlantıları engellemeyi aktifleştirin

            Bu portun eklendiğini ilgili komutu çalıştırarak gözlemleyin.
		\begin{verbatim}
		-A PARDUS-FW-MOD-BLOCK -p tcp -m multiport --dports <EklenenPort> \
		-j DROP
		-A PARDUS-OUT-MOD-BLOCK -p tcp -m multiport --dports <EklenenPort> \
		-j DROP
		\end{verbatim} 

        \item Giden bağlantıları engellemeyi iptal edin·

              Yukarıdaki satırların ilgili komutun çıktısından silindiğini doğrulayın.
        \end{enumerate}

\item Testlerin pratik bölümü:

	Her iki bilgisayarda güvenlik duvarını aktileştirin.

    	Openssh servisi kapalı ise, servis yöneticisinden başlatın.
\begin{enumerate}
    \item Gelen Bağlantıları Engelle seçeneği: 
	
	(Sabit ip'niz var ise veya aynı ağda iki adet makineniz var ise, bu adımı test edebilirsiniz, diğer durumda bu adımı geçin.)
    \begin{enumerate}
        \item Gelen Bağlantıları Engelleme seçeneğini pasifleştirin.

              Başka bir bilgisayardan kendi bilgisayırınıza bağlanmayı deneyin.

              Bu işlem için aşağıdaki komutu çalıştırın:
		\begin{verbatim}
		# ssh <sizin_bilgisayarınızın_adı>@<sabit_ip>
		\end{verbatim} 
              Bağlantının kabul edildiğini gözlemleyin.

        \item Gelen Bağlantıları Engelle seçeneğini aktifleştirin.

	Bilinen bir portu bu işlem için port olarak ekleyin.

	Bu port için ilgili bir servis var ise, bu servisi servis yöneticisinden açın.

        \begin{enumerate}
            \item Başka bir bilgisayardan kendi bilgisayarınıza uzaktan bağlanmayı deneyin.

	Bu işlem için aşağıdaki komutu çalıştırın:
	\begin{verbatim}
	# ssh <sizin_bilgisayarınızın_adı>@<sabit_ip>
	\end{verbatim}
	Bağlantıya izin verilmediğini gözlemleyin.

       \item Engellenecek olan port dışında bir port kullanarak kendi bilgisyarınıza uzaktan bağlanmayı deneyin.

	Bu işlem için aşağıdaki komutu çalıştırın:
	\begin{verbatim}
	# ssh -p <port> <sizin_bilgisayarınızın_adı>@<sabit_ip>
	\end{verbatim}
                 Bağlantının kabul edildiğini gözlemleyin.
        \end{enumerate}
    \end{enumerate}
    \item İnternet paylaşımı seçeneği: 

	(Dahili iki ethernet kartınız veya fazladan harici bir ethernet kartınız varsa bu bölümü test edebilirsiniz, yoksa bu adımı geçin.)

        \begin{enumerate}
        \item Harici veya dahili ethernet kartınızı kullanarak, kendi bilgisayarınız ile diğer bilgisayarı ethernet kablosu ile birbirine bağlayın. Eğer diğer bilgisayarın internet erişimi varsa durdurun.)

        \item Güvenlik Duvarı Yöneticisinden internet paylaşımını aktifleştirin.

        \item Birinci ethernet kartınızı internete köprü için, ikincisini ev ağınıza köprü için seçin.

              Diğer makinanın sizin makinanız üzerinden internete bağlanabildiğini gözlemleyin. (Diğer makina üzerinden ağ yöneticisi ile bunu gerçekleştirebilirsiniz.) 
        \end{enumerate}
    \item Giden Bağlantıları Engelleme seçeneği:

	(Sabit ip'niz varsa ya da aynı ağda iki makine varsa, bu adımı test edebilirsiniz, diğer durumda bu adımı atlayın.)

        \begin{enumerate}
        \item Giden bağlantı engelle seçeneğini pasifleştirin.
            \begin{enumerate}
            \item Kendi bilgisayarınızdan, bilinen bir portu kullanarak, diğer bilgisayara bağlantı oluşturmayı deneyin.

                (Uzaktaki bilgisayar statik ip'ye sahipse uzaktan bağlantı için onu kullanabilirsiniz.)

                Bu işlemi gerçekleştirebilmek için aşağıdaki komutu çalıştırın:
		\begin{verbatim}
		# ssh -p <port> <diğer_bilgisayarın_adı>@<sabit_ip>
		\end{verbatim} 
                Bağlantının kabul edildiğini gözlemleyin.

            \item Bilinen bir portu ekleyin ve giden bağlantıları engellemeyi aktifleştirin.
            	
            Kendi bilgisayarınızdan eklediğiniz portu kullanarak diğer bilgisayara uazaktan bağlantı oluşturmayı deneyin.
		\begin{verbatim}
		# ssh -p <port> <diğer_bilgisayarın_adı>@<sabit_ip>
		\end{verbatim}

                    Bağlantıya izin verilmediğini gözlemleyin. 
            \end{enumerate}
        \end{enumerate}
    \end{enumerate}
\end{enumerate}

\end{document}


