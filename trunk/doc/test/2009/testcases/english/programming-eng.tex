
\documentclass[a4paper,10pt]{article}
\usepackage[turkish]{babel}
\usepackage[utf8]{inputenc}
\usepackage[left=1cm,top=1cm,right=2cm,bottom=2cm]{geometry}

\title{Programming Companent Testcases}
\author{Semen Cirit}

\renewcommand{\labelenumi}{\arabic{enumi}.}
\renewcommand{\labelenumii}{\arabic{enumi}.\arabic{enumii}.}
\renewcommand{\labelenumiii}{\arabic{enumi}.\arabic{enumii}.\arabic{enumiii}.}
\renewcommand{\labelenumiv}{\arabic{enumi}.\arabic{enumii}.\arabic{enumiii}.\arabic{enumiv}.}

\begin{document}

\maketitle
\section{Microcontroller alt bileşeni}
\begin{enumerate}
 \item Following packages subject to installation test:
\begin{verbatim}
 avr-libc
 avrdude
 binutils-avr
 gcc-avr
\end{verbatim}
\end{enumerate}
\section{Tool sub component}
 \begin{enumerate}
 \item Following packages subject to installation test:
\begin{verbatim}
fcgi 
\end{verbatim}
\end{enumerate}

\section{vcs sub component}
\begin{enumerate}
\item mod\_dav\_svn package depended installation test.
\item After installation git package:

Run below commands. And observe that it creates Git deposu and clones it correctly.
\begin{verbatim}
 # cd ~
 # mkdir test_git
 # cd test_git
 # git init
 # cd ..
 # git clone test_git test_clone
\end{verbatim}

\item After installation subverison package:

Observe that test directory added correctly:
\begin{verbatim}
  # svn co http://svn.pardus.org.tr/uludag/trunk/test/2009/testguide/turkish/
  # cd turkish
  # svn mkdir test
  # svn st
 \end{verbatim}
Write some words to testfile and save it. observe that you fetch change difference.
\begin{verbatim}
  # vi testfile
  # svn add testfile
  # svn diff
 \end{verbatim}
\end{enumerate}

\section{Environment sub component}
\begin{enumerate}
 \item After installation eric package:
 
Open the below file with eric application and run it following Start $\rightarrow$ Run Script way. 

Observe that it runs correctly.
\begin{verbatim}
 # wget http://cekirdek.pardus.org.tr/~semen/dist/test/programming/environment/test.py
\end{verbatim}
 \item After installation below packages, change local language, open an open office application from console at same directory and observe that help file's language be same depended language.
\begin{verbatim}
eric-i18n-cs
eric-i18n-de
eric-i18n-es
eric-i18n-fr
eric-i18n-ru
eric-i18n-tr
 \end{verbatim}

for changing local language:
\begin{verbatim}
export LC_ALL= <lang_LANG>
\end{verbatim}

represented by lang\_LANG, it will be pt-BT for pt\_BT, for other languages example de\_DE

After run eric4 command at the command directory,
Daha sonra bu çalıştırdığınız komut dizininde eric4 komutunu çalıştırın, paket eğer help ile ilgili ise help dosyasının, uygulama dili ise uygulamanın sorunsuz bir şekilde istenilen dilde açıldığını gözlemleyin.

\item After installation ipython package: 

Observe that when you run below commands,
Aşağıda bulunan komutları çalıştırdığınızda, bulunduğunuz dizinde test adında bir dosya oluştuğunu ve içerisinde "test ipython" yazdığını gözlemleyin:
\begin{verbatim}
 # ipython
 a = open("test", "a")
 a.write("test ipyton")
\end{verbatim}

\item After installation drscheme package:

Open the application from Kmenu and and observe that it runs correctly.

\item After installation qt-creator package:

Open the application from Kmenu and and observe that it runs correctly
\end{enumerate}


\section{Language sub component}
\subsection{Php sub component}
\begin{enumerate}
 \item After installation php-cli and php-common packages:

After run below commands, enter http://localhost/test.php adress by firefox and observe that it paged informations about php.

\begin{verbatim}
# cd /var/www/localhost/htdocs/
# wget http://cekirdek.pardus.org.tr/~semen/dist/test/programming/language/php/test.php 
\end{verbatim}

\end{enumerate}


\subsection{Perl sub component}
\begin{enumerate}
\item After installation perl-IO-Socket-SSL package:
Download below file and open it.
\begin{verbatim}
# wget http://cekirdek.pardus.org.tr/~semen/dist/test/programming/language/perl/IO-Socket-SSL-1.26.tar.gz
\end{verbatim}

from console;
\begin{verbatim}
# cd IO-Socket-SSL-1.26/
# /usr/bin/perl5.10.0 "-MExtUtils::Command::MM" "-e" "test_harness(0,'blib/lib', 'blib/arch')" t/*.t
\end{verbatim}

run the commands and observe that returned "ok" from tests.
\item perl-Compress-Zlib paketi kurulumu sonrası:

Do programming-eng.pdf git test.

\item After installation perl-Email-MIME-Encodings package:
\begin{verbatim}
 # wget http://cekirdek.pardus.org.tr/~semen/dist/test/programming/language/perl/Email-MIME-Encodings.t
 # perl Email-MIME-Encodings.t
\end{verbatim}

observe that all results returned "ok". 

\item After installation perl-Email-MIME-Encodings package:
\begin{verbatim}
 # wget http://cekirdek.pardus.org.tr/~semen/dist/test/programming/language/perl/test_perl_Test_Simple.t
 # perl test_perl_Test_Simple.t
\end{verbatim}

observe that all results returned "ok". 


\end{enumerate}
\subsection{Python sub component}

\begin{enumerate}
\item After installation pyFltk package:

İnstall ipython package and run the following commands:
\begin{verbatim}
 # ipython
 import fltk
\end{verbatim}


\item After installation python-turboflot package:

İnstall ipython package and run the following commands:
\begin{verbatim}
 # ipython
 import turboflot
\end{verbatim}

\item After installation python-ldap package:

İnstall ipython package and run the following commands:
\begin{verbatim}
 # ipython
 import ldap
\end{verbatim}

\item After installation python-iptables package:

İnstall ipython package and run the following commands:
\begin{verbatim}
 # ipython
 import iptables
\end{verbatim}

\item After installation sympy package:

Observe that the results return ok when you run the below command.
\begin{verbatim}
 # python -c "import sympy;print sympy.test()"
\end{verbatim}


\item After installation scipy package:

Observe that the results the below command returns a result like "<nose.result.TextTestResult run=XXXX errors=0 failures=0>".
\begin{verbatim}
 # python -c "import scipy;print scipy.test()"
\end{verbatim}

\item After installation PyX package:

İnstall ipython package and run the following commands:
\begin{verbatim}
 # ipython
 import pyx
\end{verbatim}

\item After installation pyNotifier package:

İnstall ipython package and run the following commands:
\begin{verbatim}
 # ipython
 import pynotify
\end{verbatim}


\item After installation httplib2 package:

İnstall ipython package and run the following commands:
\begin{verbatim}
 # ipython
 import httplib2
\end{verbatim}

 \item After installation Django package:
\begin{itemize}
 \item run below commands:
 \begin{verbatim}
 # django-admin.py startproject test
 # cd test
 \end{verbatim}
 observe that it creates test named directory and creates below files under that directory.
  \begin{verbatim}
  __init__.py
  manage.py
  settings.py
  urls.py 
  \end{verbatim}
 \item run below commands and then enter http://localhost:8080/ adress from firefox and observe that you can connect to server.
  \begin{verbatim}
  # python manage.py runserver 8080
  \end{verbatim}
 \item  assign below database variables to DATABASE\_ENGINE DATABASE\_NAME variables in settings.py:
  \begin{verbatim}
   DATABASE_ENGINE = 'sqlite3'
   DATABASE_NAME = 'sqlite3_'   
  \end{verbatim}
\item run below command ve istemiş olduğu işlemleri sırasıyla gerçekleştirin ve sorunsuz bir şekilde Django onay sisteminin kurulduğunu gözlemleyin:
\begin{verbatim}
# python manage.py syncdb 
\end{verbatim}
\item run below command and observe that it creates poll named directory:
\begin{verbatim}
# python manage.py startapp polls 
\end{verbatim}
\item and observe that it creetes below files in the directory:
\begin{verbatim}
 __init__.py
 models.py
 views.py
\end{verbatim}

\end{itemize}

\item After installation python-memcached package:

start Apache server from service manager.

run below commands and observe that it's result returned "true".
\begin{verbatim}
 # wget http://cekirdek.pardus.org.tr/~semen/dist/test/programming/language/
python/test_python-memcache.py
 # python test_python-memcache.py
\end{verbatim}

\item After installation pygtk package: 

run below commands and observe that opened a window correctly.

\begin{verbatim}
 # wget http://cekirdek.pardus.org.tr/~semen/dist/test/desktop/toolkit/test_pango.py
 # python test_pango.py
\end{verbatim}

\item After installation mpmath package:  

install ipython package and run below commands:
\begin{verbatim}
 # ipython
 import mpmath
\end{verbatim}

\item After installation python-M2Crypto package:

install ipython package and run below commands:
\begin{verbatim}
 # ipython
 import M2Crypto
\end{verbatim}

\item After installation winpdb package:

install ipython package and run below commands:
\begin{verbatim}
 # ipython
 import winpdb
\end{verbatim}
(DeprecationWarning isn't important.)

\item After installation cython package:

install ipython package and run below commands:
\begin{verbatim}
 # ipython
 import cython
\end{verbatim}

\item After installation lxml package:  

install ipython package and run below commands:
\begin{verbatim}
 # ipython
 import lxm
\end{verbatim}
\item After installation python-RuleDispatch package:  

install ipython package and run below commands:
\begin{verbatim}
 # ipython
 import dispatch
\end{verbatim}

\item After installation python-nose package:  

install ipython package and run below commands:
\begin{verbatim}
 # ipython
 import nose
\end{verbatim}

\item After installation PyICU package:  

install ipython package and run below commands:
\begin{verbatim}
 # ipython
 import PyICU
\end{verbatim}

\item After installation python-simplejson package:  

install ipython package and run below commands:
\begin{verbatim}
 # ipython
 import simplejson
\end{verbatim}

\end{enumerate}

\subsection{Java sub component}
\begin{enumerate}
 \item After below packages installation:
\begin{verbatim}
 sun-jre
 sun-jdk
 sun-jdk-demo
 sun-jdk-samples
 sun-jdk-doc
\end{verbatim}

observe that they run below commands correctly.

\begin{verbatim}
# java -version
# wget http://cekirdek.pardus.org.tr/~semen/dist/test/programming/language/java/test.java
# javac test.java
# java test
\end{verbatim}
\end{enumerate}
\subsection{Lisp sub component}
\begin{enumerate}
 \item After installation clisp package: (don't worry about warnings.)

run below commands and observe that no error.
\begin{verbatim}
# wget http://cekirdek.pardus.org.tr/~semen/dist/test/programming/language/lisp/test_clisp.lisp 
# clisp -c test_clisp.lisp
\end{verbatim}

\end{enumerate}
\subsection{Dotnet sub component}
\begin{enumerate}
\item Following packages subject to installation test:
\begin{verbatim}
taglib-sharp 
\end{verbatim}

 \item After installation gmime, gmime-docs, gmime-sharp packages:

Observe that the following command encode the jpeg file.
\begin{verbatim}
 # wget http://cekirdek.pardus.org.tr/~semen/dist/test/multimedia/graphics/test_dcraw.jpg
 # gmime-uuencode -m test_dcraw.jpg jpeg
\end{verbatim}
 \item After installation mono package:

run below commands and observe that no error.
\begin{verbatim}
# wget http://cekirdek.pardus.org.tr/~semen/dist/test/programming/language/dotnet/test_mono.cs
# mcs test_mono.cs
# mopno test_mono.exe
\end{verbatim}
\end{enumerate}


\begin{itemize}
 \item After installation R package:

run below commands and observe that it creates a graph.
\begin{verbatim}
 # wget http://cekirdek.pardus.org.tr/~semen/dist/test/programming/language/test_R.R
 # R --vanilla --slave < test_R.R
\end{verbatim}
\item After installation R-mathlib package:

run below commands and observe that they run correctly.	
\begin{verbatim}
 # wget http://cekirdek.pardus.org.tr/~semen/dist/test/programming/language/test_r-mathlib.c
 # gcc -o test_r-matlib test_r-matlib.c -lm -lRmath
\end{verbatim}
\end{itemize}

\end{document}

