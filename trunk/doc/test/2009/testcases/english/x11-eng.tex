\documentclass[a4paper,10pt]{article}
%opening
\title{X Server Test Cases}
\author{Semen Cirit} 

\begin{document} 

\maketitle

Note: Please provide /var/log/Xorg.0.log file while reporting errors related to these packages.

\section{Server Subcomponent}

\begin{itemize}
  \item Following packages subject to installation test:
  \begin{verbatim}
  xorg-server-common
  xorg-server-devel
  xorg-server-xdmx
  xorg-server-xvfb
  \end{verbatim}

  \item Install xorg-server-xephyr and observe that a console window appears after typing
        the following command:
  \begin{verbatim}
  xinit -- /usr/bin/Xephyr :1
  \end{verbatim}

  \item Packages to be installed and tests to be applied are given below.
  \begin{verbatim}
  zorg    
  xorg-server
  libdrm
  mesa
  \end{verbatim}


  \begin{enumerate}
  \item Basic functions
    \begin{enumerate}
    \item Reboot your system and observe that the graphical login starts succesfully.
    \item Repeat the test with the option "Probe Again" selected in "Graphical Display (F4)" menu of boot loader.
    \item Ensure that xterm package is installed and type the following command:
      \begin{verbatim}
	xinit -- :2
      \end{verbatim}
      Observe that a terminal window is started in a new X screen. Move mouse cursor over the terminal window and type "exit" to return.
    \item Move and resize an application window. Observe that these operations are performed smoothly and fast.
    \end{enumerate}

  \item 3D support:
    \begin{enumerate}
    \item Run "glxgears" and "glxgears -fullscreen" in a console window. Observe that three rotating gears appear without a problem.
    \end{enumerate}

  \item VT switch:
    \begin{enumerate}
    \item Press \texttt{Ctrl+Alt+F1} in the graphical screen. Observe that it switches to TTY console without a problem.
    \item Use \texttt{Ctrl+Alt+F7} buttons to go back. Observe that no problems appear with this switch.
    \end{enumerate}

  \item Resolution test:
    \begin{enumerate}
    \item Run Settings $\rightarrow$ Display Settings (for 2008, Display Manager in TASMA).
    \item On the screens module, choose a resolution for your output device.
    \item Relogin and check that the resolution you selected and the resolution appears in the output of "xrandr" command are same.
          There will be a * symbol near the current resolution.
    \end{enumerate}
 
  \item Video test:
    \begin{enumerate}
    \item Open a video file using one of the media players and observe that it plays without a problem.
    \end{enumerate}
 
  \item (For 2009) Efect test:
    \begin{enumerate}
    \item Toggle the effects using the module in System Settings $\rightarrow$ Desktop. Observe that no problems appear.
    \end{enumerate}
  \end{enumerate}
\end{itemize}

\section{Driver Subcomponent}
\begin{enumerate}
  \item After installation virtualbox-guest-utils package:

Perform the tests given in hardware-tr.pdf.

  \item After installation xorg-input-synaptics package:

You can test this package if the following command produces an output.
  \begin{verbatim}
   # grep -i synap /proc/bus/input/devices
  \end{verbatim}

Reboot your computer and observe that your touchpad device works correctly.

\item After installation  xorg-input-vmmouse package:

You can test this package if you use a Pardus guest machine on vmware.

Test the integration works correctly in the guest machine.

\item After installation  xorg-input-wacom package:

You can test this package if you have a Wacom tablet.

Plug your tablet and observe that it works correctly.

\item The following packages will be tested in the same way. You will test the driver which supports your video card.

For example: If you have an Alliance Promotion video card, you will test xorg-video-apm package.

You can see the detailed info about a package with the command below:
\begin{verbatim}
 # pisi info <package name>
\end{verbatim}

Ekran kartınızın ne olduğuna dair bilgiye aşağıda bulunan komut ile ulaşabilirsiniz.
In order to learn the video card you are using, see the output of the command below:
\begin{verbatim}
 # lspci
\end{verbatim}

\begin{verbatim}
  xorg-video-apm 
  xorg-video-ast
  xorg-video-cirru
  xorg-video-fbdev (This driver can be used on any video card, so that it can be tested.)
  xorg-video-glint
  xorg-video-i128
  xorg-video-i740
  xorg-video-intel
  xorg-video-mach64
  xorg-video-mga
  xorg-video-neomagic
  xorg-video-r128
  xorg-video-radeon
  xorg-video-radeonhd
  xorg-video-s3
  xorg-video-s3virge
  xorg-video-savage
  xorg-video-siliconmotion
  xorg-video-sis
  xorg-video-sisusb
  xorg-video-tdfx
  xorg-video-trident,
  xorg-video-vesa (All testers can test this driver. Reboot in safe mode and perform the tests
                   just like the other drivers.)
  xorg-video-vmware (Test this if you have a VmWare guest machine)
  xorg-video-voodoo
  xorg-video-chips
\end{verbatim}

\begin{itemize}
  \item Reboot your computer and observe that the correct resolution is used.

  Observe that the drawing operations are not slow.
  \item Play the following video with one of the players like mplayer, kaffeing, etc. Observe that no problem exists.
  \begin{verbatim}
  # wget http://cekirdek.pardus.org.tr/~semen/dist/test/multimedia/video/cokluortam/DVD.mpg
  \end{verbatim}
\end{itemize}

\end{enumerate}

\section{Terminal Subcomponent}
\begin{enumerate}
  \item After installation  rxvt-unicode package:

  Observe that an X console window appears.
  \begin{verbatim}
  # urxvt
  \end{verbatim}

  The first command below starts the server. In a console window start the server and type the second command on another console.
Observe that an X console window appears.
  \begin{verbatim}
  # urxvtd (start the server)
  # urxvtc
  \end{verbatim}

  \item After installation  xterm package:

  Observe that the following command open a new console window.
  \begin{verbatim}
  # xterm
  \end{verbatim}
\end{enumerate}
\end{document}
