\documentclass[a4paper,10pt]{article}
 \usepackage[left=1cm,top=1cm,right=2cm,bottom=2 cm]{geometry}
\title{Firewall Manager Test Cases}
\author{Semen Cirit}

\renewcommand{\labelenumi}{\arabic{enumi}.}
\renewcommand{\labelenumii}{\arabic{enumi}.\arabic{enumii}.}
\renewcommand{\labelenumiii}{\arabic{enumi}.\arabic{enumii}.\arabic{enumiii}.}
\renewcommand{\labelenumiv}{\arabic{enumi}.\arabic{enumii}.\arabic{enumiii}.\arabic{enumiv}.}

\begin{document}

\maketitle

\begin{enumerate}
    \item Try to open firewall manager from system settings.
    \item Try to open firewall manager from Kmenu.
    \item Try to stop firewall service from the start/stop button of firewall manager.

    Execute the following command:

\begin{verbatim}
    # service list
\end{verbatim} 

    Observe that iptables service gets off.

    \item Try to start service from the start/stop button of firewall manager.

        Execute the following command:

\begin{verbatim}
    # service list
\end{verbatim}

       Observe that iptables service gets on.

\item Push the configuration button and try these actions for options: Block incoming connections or Block outgoing connections.
    \begin{enumerate}
        \item Adding  port
        \item Removing port
        \item Taking up a port
        \item Taking down a port
    \end{enumerate}

\textbf{Note:} Observe the outputs of this command for each situation.

Execute the following command:

\begin{verbatim}
    # iptables --list-rules
\end{verbatim} 

\item Block incoming connections:

\begin{enumerate}
     \item After adding a port from firewall manager.
        \begin{enumerate}
        \item Enable block incoming connections:
            	Observe that the output of the above command contains these: 
\begin{verbatim}
-A PARDUS-IN-MOD-SERVING -p tcp -m multiport --dports <addedPORT> \
    -j ACCEPT
-A PARDUS-IN-MOD-SERVING -p tcp -m multiport --dports 0:1024 \
    -m tcp --tcp-flags FIN,SYN,RST,ACK SYN -j REJECT --reject-with \
    icmp-port-unreachable
-A PARDUS-IN-MOD-SERVING -p udp -m multiport --dports 0:1024 \
    -j REJECT --reject-with icmp-port-unreachable
\end{verbatim} 
        \item Disable block incoming connections

              Observe that the above lines are removed from the output of the below command.
        \end{enumerate}
    \item Without adding a port

        Observe that the output of the above command contains these:

\begin{verbatim}
-A PARDUS-IN-MOD-SERVING -p tcp -m multiport --dports 0:1024 \
    -m tcp --tcp-flags FIN,SYN,RST,ACK SYN -j REJECT --reject-with \
    icmp-port-unreachable
-A PARDUS-IN-MOD-SERVING -p udp -m multiport --dports 0:1024 \
    -j REJECT --reject-with icmp-port-unreachable
\end{verbatim} 


\item Internet sharing
    \begin{enumerate}
    \item Enable Internet sharing
    	(If you have two internal ethernet card or an extra external ethernet card, you can test this part, if not skip the following step.
        \begin{enumerate}

        \item Select different values for gate to internet and gate to home network
            	Observe that the above command output contains these:
\begin{verbatim}
-A PARDUS-FW-MOD-SHARING -i <input> -o <output> -m state \
    --state ESTABLISHED,RELATED -j ACCEPT
-A PARDUS-FW-MOD-SHARING -i <output> -o <input> -j ACCEPT
-t nat -A PARDUS-POST-MOD-SHARING -o <input> -j MASQUERADE
\end{verbatim} 
        \item Give same values

    	Observe that the above lines are removed from the output of the below command.
        \end{enumerate}
    \end{enumerate}

\item Block outgoing connections 
\begin{enumerate}
    \item After adding a port from firewall manager.
        \begin{enumerate}
        \item Enable block outgoing connections

            Observe that the port is added

            There should be lines like below at output of the command:

\begin{verbatim}
-A PARDUS-FW-MOD-BLOCK -p tcp -m multiport --dports <addedPORT> \
    -j DROP
-A PARDUS-OUT-MOD-BLOCK -p tcp -m multiport --dports <addedPORT> \
    -j DROP
\end{verbatim} 

        \item Disable block outgoing connections·

              Observe that the above lines are removed from the output of the below command.
        \end{enumerate}
    \end{enumerate}
\end{enumerate}
\item Pratical part of tests

	First activate firewall on both two computers.	

    	If the service openssh was disabled, start it from service manager.
\begin{enumerate}
    \item Block incoming connections: 

	
		(If you have a static ip or your two machine is in the same network, you can test this case, if not skip it)
    \begin{enumerate}
        \item Disable block incoming connections

              Try to make a remote connection from an other computer to your computer.

              Execute the following command:
		\begin{verbatim}
		# ssh <your_computer_name>@<static_ip>
		\end{verbatim} 
              Observe that the connection is being accepted.

        \item Try to add a well known port and enable block incoming connections.
            Open the related service of this port from service-manager.

        \begin{enumerate}
            \item Try to make a remote connection from an other computer to your computer.
	\begin{verbatim}
	# ssh <your_computer_name>@<static_ip>
	\end{verbatim}
	Observe that the connection refused.

            \item Try to make a remote connection using the added port from an other computer to your computer.
	\begin{verbatim}
	# ssh -p <port> <your_computer_name>@<static_ip>
	\end{verbatim}
                 Observe that the connection is accepted.
        \end{enumerate}
    \end{enumerate}
    \item For internet sharing: 

	(If you have two internal ethernet card or an extra external ethernet card and an extra computer, you can test this part, if not skip it.)
        \begin{enumerate}
        \item Plug an extra external or internal ethernet card to your computer. And plug an ethernet cable between your computer and the other. ( And if the other computer have an internet acces please stop it. )

        \item Enable internet sharing from firewall-manager.

        \item Choose the first ethernet card for gate to internet, and the second for gate to home network.

              Now try to connect to connect internet over your computer and observe it.
        \end{enumerate}
    \item Block outgoing connections:

	(If you have a static ip or your two machine is in the same network, you can test this case, if not skip it)
        \begin{enumerate}
        \item Disable access restriction
            \begin{enumerate}
            \item Try to make a remote connection with using a well known port from your computer to the other remote computer.
                If the remote computer has a static ip you can use it for remote connection. 

                Execute the following command:
		\begin{verbatim}
		# ssh -p <port> <remote_computer_name>@<static_ip>
		\end{verbatim} 
                Observe that the connection accepted.

            \item Try to add a well known port and enable block outgoing connections.
            	
                  Try to make a remote connection using added port from your computer to the other computer.
\begin{verbatim}
    # ssh -p <port> <remote_computer_name>@<static_ip>
\end{verbatim}

                    Observe that the connection refused. 
            \end{enumerate}
        \end{enumerate}
    \end{enumerate}
\end{enumerate}

\end{document}


